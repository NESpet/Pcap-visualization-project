\documentclass{beamer}

\usepackage[francais]{babel}
\usepackage[T1]{fontenc}
\usepackage[utf8]{inputenc}
\usepackage{beamerthemecs}
\usepackage{beamerouterthemecs}
\usepackage{beamerfontthemecs}
\usepackage{beamerinnerthemecs}
\usepackage{beamercolorthemecs}

\usetheme{cs}
\useoutertheme{cs}
\usefonttheme{cs}
\useinnertheme{cs}
\usecolortheme{cs}

\title[Visualisation de traces réseaux]{Visualisation de traces réseaux}
\author{\textbf{Thibault \textsc{Lengagne}et Nicolas \textsc{Ngô-Maï}}}
\institute{Centrale Supélec - Campus de Rennes}

\begin{document}

  \begin{frame}
    \titlepage
  \end{frame}
  
  \AtBeginSection[] {
    \begin{frame}
      \frametitle{Plan}
      \tableofcontents[currentsection, hideothersubsections, pausesubsections]
    \end{frame} 
  }

 \section{Introduction}
  \begin{frame}
   \frametitle{Réseaux Zigbee}
   \begin{itemize}
    \item Histoire breve
    \item Intérêt, application (domotique)
    \item Droit, license, implémentation
    \item Avenir

   \end{itemize}
  \end{frame}
  
  \begin{frame}
   \frametitle{Fonctions principales}
   Depuis \textit{Zigbee}
   \begin{itemize}
    \item 
    \item 
    \item 
    \item 
   \end{itemize}
  \end{frame}

  \begin{frame}
     \frametitle{Stratégie}
    \begin{itemize}
     \item 
     \item 
    \end{itemize}
    \begin{alertblock}{Example}
      Example
    \end{alertblock}
  \end{frame}
  
 \section{La norme IEEE 802.15.4}
  \begin{frame}
    \frametitle{Zigbee et la norme 802.15.4}
    Le protocole Zigbee utilise ce protocole comme cadre de fonctionnement :
      \begin{center}
       \includegraphics[scale=0.4]{OSI-Zigbee.png}
      \end{center}  
  \end{frame}

  \begin{frame}
    \frametitle{Détails}
    Protocole de communication de la famille des LR WPAN
    \begin{itemize}
      \item Faible consommation
      \item Faible portée
      \item Faible débit
    \end{itemize}
  \end{frame}

  \begin{frame}
    \frametitle{Comparatif}
    \begin{block}{Schema comparatif des différents protocole sans fil}
      \begin{center}
       \includegraphics[scale=0.3]{Range.png}
      \end{center} 
    \end{block}
  \end{frame}
  
  \begin{frame}
    \frametitle{La couche physique}
    Contient l'émetteur/récepteur radio, avec un mécanisme de contrôle de qualité du signal et CCA
    \begin{block}{Débit}
      \begin{center}
       \includegraphics[scale=0.25]{Vitesse-Zigbee.png}
      \end{center}
    \end{block}
  \end{frame}
  
  \begin{frame}
    \frametitle{La couche d'accès au medium (MAC)}
    \begin{block}{Rôle des éléments du réseau}
      \begin{itemize}
        \item Le coordinateur (ZC) est le noeud principal, il est unique
        \item Les FFD ou routeurs gèrent le routage et les terminaux
        \item Les RFD ou terminaux sont de simple capteurs aux extremités du réseau
      \end{itemize}
    \end{block}
  \end{frame}
  
  \begin{frame}
    \frametitle{La couche d'accès au medium (MAC)}
    \begin{block}{Format de trame}
      \begin{itemize}
        \item En-tête (contrôle de trame, numéro de séquence, adressage)
        \item Données
        \item Pied (CRC)
        \begin{center}
         \includegraphics[scale=0.5]{Couche-MAC.png}
        \end{center} 
      \end{itemize}
    \end{block}
    \begin{block}{Il existe deux modes de fonctionnement}
      \begin{itemize}
        \item Le mode non-coordonnée
        \item Le mode coordonnée, ou balisé
      \end{itemize}
    \end{block}
  \end{frame}
  
  \begin{frame}
    \frametitle{La couche d'accès au medium (MAC)}
    \begin{block}{Le mode non-Coordonnée}
      \begin{itemize}
        \item Pas d'emission de \textit{beacon}
        \item Fonctionnement CSMA/CA pour gérer les collisions
        \item Le coordinateur est éveillé en permanence
      \end{itemize}
    \end{block}
  \end{frame}

  \begin{frame}
    \frametitle{La sous-couche d'accès au medium (MAC)}
    \begin{block}{Le mode Coordonnée}
      Le coordinateur diffuse périodiquement des \textit{beacon}. Tous les dispositifs sont informés de :
      \begin{itemize}
        \item La durée de la \textit{superframe} et quand ils peuvent transmettre des données en CSMA/CA
        \item A partir de quel moment le coordinateur rentre en hibernation et pour quelle durée
	\begin{center}
	\includegraphics[scale=0.4]{Supertrame.png}
	\end{center} 
      \end{itemize}
    \end{block}
  \end{frame}

  \begin{frame}
    \frametitle{La sous-couche de convergence (LLC)}
    \begin{itemize}
      \item Vérification de l'intégrité des données reçues
      \item Contrôle de flux, afin d'éviter la saturation
      \item La convergence d'adressage (correspondance entre les couches 2 et 3 du modèle OSI, gestion du boradcast et en multicast)
    \end{itemize}
  \end{frame}

  \section{Les couches 4 à 7}
  \begin{frame}
    \frametitle{Présentation}
    \begin{itemize}
      \item 
      \item 
    \end{itemize}
    Example : 
    \begin{itemize}
      \item 
      \item 
    \end{itemize}
  \end{frame}

  \begin{frame}
    \frametitle{Adressage}
    \begin{itemize}
      \item Zigbee propose un algorithme de distribution d'adresses atomatique et décentralisé
      \item 
    \end{itemize}
    Example : 
    \begin{itemize}
      \item 
      \item 
    \end{itemize}
  \end{frame}

  \section{Conclusion}
  \begin{frame}
    \begin{center}
      Merci de votre attention !
    \end{center}
  \end{frame}

\end{document}
